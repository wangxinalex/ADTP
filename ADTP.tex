\documentclass[twocolumn]{article}
\usepackage{amsmath}
\usepackage{epsfig}
\usepackage{psfig}
\usepackage{graphicx}
\usepackage{url}
\usepackage{times}
\usepackage{listings}
\usepackage{multirow}
\usepackage[colorlinks,linkcolor=red,anchorcolor=blue,citecolor=green]{hyperref}
\setlength{\columnsep}{22pt}
\newcommand{\MLR}{Multiple Linear Regression}
\newcommand{\mlr}{multiple linear regression}
\newcommand{\Mlr}{Multiple linear regression}
\lstdefinestyle{rlanguage}{
  belowcaptionskip=1\baselineskip,
  breaklines=true,
  frame=L,
  xleftmargin=\parindent,
  language=R,
  showstringspaces=false,
  basicstyle=\footnotesize\ttfamily,
  numbers=left,
  numbersep=4pt,
  keywordstyle=\bfseries\color{red},
  commentstyle=\itshape\color{gray},
  identifierstyle=\color{blue},
  stringstyle=\color{orange},
}
\begin{document}
\lstset{escapechar=@,style=rlanguage}
\title{Linear Regression Analysis of Bike Sharing System }

\author{Wang Xin\\ \vspace{0.5cm}\\
 10302010023}

\date{}
\maketitle
\begin{abstract}
In this report, we describe a brief process of multiple linear regression for a bike sharing system. We exploit the R language to estimate a linear model of bike sharing counts to multiple predictors and detailed analyses will be given in this report.
The final model suggests a significant relationship between bike sharing counts and predictors given in the input data file like whether conditions and workdays. It is also noteworthy that the resultant model is not the unique answer since the relationship among predictors and outcomes could be explained in more than one mathematical expressions.
\end{abstract}

\section{Introduction}
\label{sec:intro}
Bike sharing systems are new generation of traditional bike rentals where whole process from membership, rental and return
back has become automatic. Through these systems, user is able to easily rent a bike from a particular position and return
back at another position. Currently, there are about over 500 bike-sharing programs around the world which is composed of
over 500 thousands bicycles. Today, there exists great interest in these systems due to their important role in traffic,
environmental and health issues.

Apart from interesting real world applications of bike sharing systems, the characteristics of data being generated by
these systems make them attractive for the research. Opposed to other transport services such as bus or subway, the duration
of travel, departure and arrival position is explicitly recorded in these systems. This feature turns bike sharing system into
a virtual sensor network that can be used for sensing mobility in the city. Hence, it is expected that most of important
events in the city could be detected via monitoring these data.

In this report we will attempt to exploit {\mlr} using R language to construct a linear model. The responses consists of the count of total rental bikes including casual users and registered users and the predictors contains time and weather-related metrics. Based on the model, we attempt to predict the rental count using the given predictor values and provide some suggestions towards the plan of new expansion zones.

The rest of this paper is organized as follows. Section~\ref{sec:technique} provides a small description of the technique used (\mlr), along with its strengths and weaknesses.
Section~\ref{sec:database} relates the data fields appeared in the dataset file provided. Section~\ref{sec:outcome} analyzes the data manipulation as well as the graphs and tables. Section~\ref{sec:model} describes the final suggested model and analyze its performance. Finally, Section~\ref{apd:commands} in appendix presents all R commands used and Section~\ref{apd:pre} gives the preprocessing script. 

\section{Analysis of Techniques}
\label{sec:technique}

\section{Analysis of Database}
\label{sec:database} 
Bike-sharing rental process is highly correlated to the environmental and seasonal settings. For instance, weather conditions,
precipitation, day of week, season, hour of the day, etc. can affect the rental behaviors. The core data set is related to
the two-year historical log corresponding to years 2011 and 2012 from Capital Bikeshare system, Washington D.C., USA which is
publicly available in http://capitalbikeshare.com/system-data. We aggregated the data on two hourly and daily basis and then
extracted and added the corresponding weather and seasonal information. Weather information are extracted from http://www.freemeteo.com.

The dataset provided in file \texttt{hour.csv} consists of data fields shown in Table~\ref{tab:fields}.

\begin{table}
\small
\caption{Data fields in hour.csv}
\centering
\label{tab:fields}
\begin{tabular}{|l|l|}
  \hline
  \textbf{Fields} & \textbf{Meaning} \\\hline
  instant&Record index\\\hline
  dteday&Date\\\hline
  season & Season (1:springer, 2:summer, 3:fall, 4:winter)\\\hline
    yr & Year (0: 2011, 1:2012)\\\hline
    mnth & Month ( 1 to 12)\\\hline
	hr & Hour (0 to 23)\\\hline
	holiday & Weather day is holiday or not \\\hline
	weekday & Day of the week\\\hline
    workingday & If day is neither weekend nor holiday is 1, otherwise is 0\\\hline
	\multirow{4}*{weathersit} &
		 1: Clear, Few clouds, Partly cloudy, Partly cloudy\\
		& 2: Mist + Cloudy, Mist + Broken clouds\\
		& 3: Light Snow, Light Rain + Thunderstorm \\
		& 4: Heavy Rain + Ice Pallets + Thunderstorm + Mist\\\hline
	temp & Normalized temperature in Celsius\\\hline
	atemp& Normalized feeling temperature in Celsius\\\hline
	hum& Normalized humidity. \\\hline
	windspeed& Normalized wind speed\\\hline
	casual& Count of casual users\\\hline
	registered& Count of registered users\\\hline
	cnt& Count of total rental bikes including casual and registered\\
  \hline
\end{tabular}

\end{table}

In the previous 17 data fields, the last three fields \emph{casual}, \emph{registered} and \emph{cnt} are the responses of our model. Most of the rest fields are quantitative predictors, however there are some predictors that should be treated as qualified predictors like \emph{holiday} or \emph{weekday}; such data fields need to be preprocessed so that R language could take them as qualified predictors and create dummy variables automatically. As to the temporal fields such as \emph{season, hr, mnth}, we will initially take them as discrete variables and explore the method of transforming them into continuous variable. 

\section{Analysis of Outcomes}
\label{sec:outcome}
\subsection{Data Manipulation}
\subsubsection{Preprocessing}
Before we get down to read all data into our workbench in R, we need to preprocess the file format so as to delete some redundant data and adjust the format of some fields. Generally, out preprocessing includes the following steps:
\begin{itemize}
  \item \textbf{Delete Redundant Data}. For instances, the field \texttt{instant} in  Table~\ref{tab:fields} only records the indexes of data entries, which should be eliminated from the data manipulation.
  \item \textbf{Format Transformation}. The data format of some fields should be transformed, especially some \emph{qualitative} fields such as \texttt{weathersit} and \texttt{holiday}, so that R language could recognize them as qualitative variables and create dummy variables. 
\end{itemize}

We use a simple python script to preprocess the input file, which is provided in Section~\ref{apd:pre} in appendix. After preprocessing, there are totally 15 variables left in the model, the deleted ones are labeled as {\color{red}red} and the transformed ones are labeled as {\color{green}green} in Table~\ref{tab:fields}.

\subsubsection{Fit all Linear Variables}

In the remaining 15 variables, we first fit all predictors in linear form and the response is \texttt{cnt}:
\begin{lstlisting}[style=rlanguage]
>lm.fit_all = lm(cnt~.-registered-casual,data=bicycle)
>summary{lm.fit_all}

Call:
lm(formula = cnt ~ . - registered - casual, data = bicycle)

Residuals:
    Min      1Q  Median      3Q     Max
-396.57  -60.52   -7.94   51.34  440.09

Coefficients:
               Estimate Std. Error t value Pr(>|t|)
(Intercept)    -50.9441     8.9203  -5.711 1.14e-08 ***
seasonSpring   -32.0083     5.7453  -5.571 2.57e-08 ***
seasonSummer     6.2138     4.9737   1.249  0.21156
seasonWinter    35.9871     5.1574   6.978 3.11e-12 ***
.....
---
Signif. codes:  0 ‘***’ 0.001 ‘**’ 0.01 ‘*’ 0.05 ‘.’ 0.1 ‘ ’ 1

Residual standard error: 101.7 on 17330 degrees of freedom
Multiple R-squared:  0.6863,	Adjusted R-squared:  0.6855
F-statistic:   790 on 48 and 17330 DF,  p-value: < 2.2e-16
\end{lstlisting}
From the summary we could see that R language has calculated the slope and intercept for all input linear predictors, besides, it also creates dummy variables for qualitative predictors like \texttt{seasonSpring}, \texttt{seasonSummer}. For detailed information about dummy variables, we can use \texttt{Contrasts} function.

\begin{lstlisting}[style=rlanguage]
> contrasts(season)
       Spring Summer Winter
Autumn      0      0      0
Spring      1      0      0
Summer      0      1      0
Winter      0      0      1
\end{lstlisting}

The result shows that $F-statistic = 729.1$ and $p-value < 2.2e-16$ thus there \textbf{is} a relationship between the predictors and the response \texttt{cnt}. 



\subsection{Graphs and Tables} 

\section{Suggested Model}
\label{sec:model}

Finally, we exploit {\mlr} to generate regression model for three responses \texttt{casual,registered,cnt}. The results are shown as follows.

\subsection{Cnt}
\begin{lstlisting}[style=rlanguage]
Call:
lm(formula = cnt ~ season + yr + mnth + hr + holiday + weekday +
    weathersit + (temp + hum + windspeed)^2 + poly(hum, 4) +
    I(windspeed^2), data = bicycle)

Residuals:
    Min      1Q  Median      3Q     Max
-366.40  -59.41   -5.70   50.19  424.33

Coefficients: (1 not defined because of singularities)
               Estimate Std. Error t value Pr(>|t|)
(Intercept)    -150.337     13.187 -11.400  < 2e-16 ***
seasonSpring    -31.664      5.699  -5.556 2.80e-08 ***
seasonSummer      8.979      4.925   1.823  0.06831 .
seasonWinter     35.209      5.115   6.883 6.06e-12 ***
yry1             84.671      1.559  54.324  < 2e-16 ***
...
weathersitMist   -9.529      1.909  -4.992 6.02e-07 ***
weathersitRain  -83.097     58.292  -1.426  0.15403
weathersitSnow  -60.884      3.387 -17.976  < 2e-16 ***
temp            380.473     19.070  19.951  < 2e-16 ***
hum             114.093     14.960   7.626 2.54e-14 ***
windspeed        85.546     36.478   2.345  0.01903 *
poly(hum, 4)1        NA         NA      NA       NA
poly(hum, 4)2  -989.456    113.454  -8.721  < 2e-16 ***
poly(hum, 4)3   289.732    103.030   2.812  0.00493 **
poly(hum, 4)4   274.210    104.049   2.635  0.00841 **
I(windspeed^2) -215.503     36.334  -5.931 3.07e-09 ***
temp:hum       -340.669     24.058 -14.161  < 2e-16 ***
temp:windspeed  190.288     35.142   5.415 6.21e-08 ***
hum:windspeed  -181.499     35.112  -5.169 2.38e-07 ***
---
Signif. codes:  0 ‘***’ 0.001 ‘**’ 0.01 ‘*’ 0.05 ‘.’ 0.1 ‘ ’ 1

Residual standard error: 100.6 on 17320 degrees of freedom
Multiple R-squared:  0.6933,	Adjusted R-squared:  0.6923
F-statistic: 675.1 on 58 and 17320 DF,  p-value: < 2.2e-16
\end{lstlisting}

\subsection{Casual}
\begin{lstlisting}[style = rlanguage]
> lm.fit_fin = lm(casual~season+yr+mnth+hr+holiday+weekday+weathersit+(temp+hum+windspeed)^2+poly(hum,4)+I(windspeed^2),data=bicycle)
> summary(lm.fit_fin)

Call:
lm(formula = casual ~ season + yr + mnth + hr + holiday + weekday +
    weathersit + (temp + hum + windspeed)^2 + poly(hum, 4) +
    I(windspeed^2), data = bicycle)

Residuals:
    Min      1Q  Median      3Q     Max
-91.485 -18.393  -3.001  12.852 251.042

Coefficients: (1 not defined because of singularities)
                Estimate Std. Error t value Pr(>|t|)
(Intercept)     -36.9387     4.0705  -9.075  < 2e-16 ***
seasonSpring     -1.5635     1.7591  -0.889 0.374127
seasonSummer      9.0139     1.5203   5.929 3.10e-09 ***
seasonWinter      0.4354     1.5790   0.276 0.782745
yry1             11.6976     0.4811  24.314  < 2e-16 ***
...
temp            158.5548     5.8865  26.935  < 2e-16 ***
hum              50.0185     4.6179  10.831  < 2e-16 ***
windspeed        -5.3359    11.2598  -0.474 0.635584
poly(hum, 4)1         NA         NA      NA       NA
poly(hum, 4)2  -218.2046    35.0205  -6.231 4.75e-10 ***
poly(hum, 4)3    82.0113    31.8030   2.579 0.009925 **
poly(hum, 4)4   101.5730    32.1174   3.163 0.001567 **
I(windspeed^2)  -53.8757    11.2155  -4.804 1.57e-06 ***
temp:hum       -152.3517     7.4260 -20.516  < 2e-16 ***
temp:windspeed   72.3110    10.8474   6.666 2.70e-11 ***
hum:windspeed   -31.4267    10.8382  -2.900 0.003741 **
---
Signif. codes:  0 ‘***’ 0.001 ‘**’ 0.01 ‘*’ 0.05 ‘.’ 0.1 ‘ ’ 1

Residual standard error: 31.06 on 17320 degrees of freedom
Multiple R-squared:  0.6045,	Adjusted R-squared:  0.6032
F-statistic: 456.5 on 58 and 17320 DF,  p-value: < 2.2e-16
\end{lstlisting}

\subsection{Registered}
\begin{lstlisting}[style=rlanguage]
> lm.fit_fin = lm(registered~season+yr+mnth+hr+holiday+weekday+weathersit+(temp+hum+windspeed)^2+poly(hum,4)+I(windspeed^2),data=bicycle)
> summary(lm.fit_fin)

Call:
lm(formula = registered ~ season + yr + mnth + hr + holiday +
    weekday + weathersit + (temp + hum + windspeed)^2 + poly(hum,
    4) + I(windspeed^2), data = bicycle)

Residuals:
    Min      1Q  Median      3Q     Max
-346.74  -48.33   -4.84   45.05  409.22

Coefficients: (1 not defined because of singularities)
                 Estimate Std. Error t value Pr(>|t|)
(Intercept)    -113.39857   11.12982 -10.189  < 2e-16 ***
seasonSpring    -30.10083    4.80975  -6.258 3.98e-10 ***
seasonSummer     -0.03502    4.15677  -0.008 0.993279
seasonWinter     34.77392    4.31732   8.055 8.49e-16 ***
yry1             72.97351    1.31547  55.473  < 2e-16 ***
...
I(windspeed^2) -161.62695   30.66584  -5.271 1.38e-07 ***
temp:hum       -188.31749   20.30450  -9.275  < 2e-16 ***
temp:windspeed  117.97659   29.65931   3.978 6.99e-05 ***
hum:windspeed  -150.07255   29.63427  -5.064 4.14e-07 ***
---
Signif. codes:  0 ‘***’ 0.001 ‘**’ 0.01 ‘*’ 0.05 ‘.’ 0.1 ‘ ’ 1

Residual standard error: 84.92 on 17320 degrees of freedom
Multiple R-squared:  0.6863,	Adjusted R-squared:  0.6852
F-statistic: 653.2 on 58 and 17320 DF,  p-value: < 2.2e-16
\end{lstlisting}

\subsection{Graphs}
Here we present the residual graph for all the three predictors. From the following three graphs we could observe that there are no explicit patterns in \texttt{Residual vs Fitted} graph, indicating that a linear model could describe the relationship between predictors and response well. Moreover, most points in \texttt{Normal Q-Q} graph locates near the line with the slope of 45 degree thus the standardized residuals obeys to a normal distribution with $\sigma = 0$, which means that the normality hypothesis holds.

\begin{figure}[ht]
  \centering
  \includegraphics[width=\linewidth]{pic/cnt.png}\\
  \caption{Residual Graphs about Response \texttt{cnt}}\label{fig:cnt}
\end{figure}

\begin{figure}[ht]
  \centering
  \includegraphics[width=\linewidth]{pic/casual.png}\\
  \caption{Residual Graphs about Response \texttt{casual}}\label{fig:casual}
\end{figure}

\begin{figure}[ht]
  \centering
  \includegraphics[width=\linewidth]{pic/registered.png}\\
  \caption{Residual Graphs about Response \texttt{registered}}\label{fig:registered}
\end{figure}

\subsection{Questions}
Now we could answer the seven important questions.
\begin{enumerate}
  \item \textit{Is there a relationship between predictors and responses?}\\
   We first set the null hypothesis, $H_0 = \beta_0 = \beta_1 = ... = \beta_p = 0$. Then we compute the F-statistic, which tells us if we should reject the null hypothesis:
      \\$F-statistic$ = 456.5 on 58 and 17320 DF  \\$p-value <$ 2.2e-16\\
   A high F-statistic and low p-value indicate clear evidence between predictors and responses.

  \item \textit{How strong is the relationship between predictors and responses?}\\
      We compute two measures: RSE and $R^2$.\\
       RSE (Residual Standard Error): standard deviation of the response from the population regression line;
       \begin{equation}\label{equ:rse}
         RSE = \sqrt{\frac{1}{n-p-1}RSS}
       \end{equation}
        $R^2$: percentage of variability in the response that is explained by the predictors.
        \begin{equation}\label{equ:rsq}
        R^2 = \frac{TSS-RSS}{TSS} = 1-\frac{RSS}{TSS}
       \end{equation}
       Taking \texttt{registered} as an example, RSE = 84.92 and $R^2$ = 0.6852, thus almost 70\% of variance in \texttt{registered} is explained by the predictors.
  \item \textit{Which predictors contribute to responses?}\\
      From the summary  in Section~\ref{sec:fitall} we could find that after data transformation, all predictors except \texttt{atemp} and \texttt{workingday} contribute to the responses.
  \item \textit{ How accurately can we estimate the effect of each predictor on responses?}\\
      We can use \texttt{confint} function to construct 95\% confidence intervals $(2 \times SE(\hat{\beta_i})$ for each $X_i$).
      \begin{lstlisting}[style=rlanguage]
> confint(lm.fit_reg)
                       2.5 %        97.5 %
(Intercept)    -135.21414187  -91.58299531
seasonSpring    -39.52842681  -20.67323457
seasonSummer     -8.18271186    8.11267767
seasonWinter     26.31153938   43.23629561
...
poly(hum, 4)4     0.50778077  344.76708025
I(windspeed^2) -221.73508938 -101.51881134
temp:hum       -228.11636613 -148.51860843
temp:windspeed   59.84134008  176.11184295
hum:windspeed  -208.15871196  -91.98639067
\end{lstlisting}
  \item \textit{How accurately can we predict future responses?}\\
      To predict an \emph{individual response}, $Y = f(X) + \epsilon$, we can calculate prediction interval.
\begin{lstlisting}[style = rlanguage]
>predict(lm.fit_reg,data.frame(season="Spring",yr = "y0",mnth ="m2", holiday = "No", hr = "h10",weekday = "w5", weathersit="Clear",temp = 0.24, atemp = 0.2879,hum=0.80,windspeed = 0.2),interval="prediction")
      fit       lwr      upr
1 40.1079 -126.6063 206.8221
\end{lstlisting}
Thus 95\% \textbf{Prediction Interval} is [-126.6063, 206.8221].\\
To predict an \emph{average response}, $f(X)$, we can calculate confidence interval.
\begin{lstlisting}[style = rlanguage]
> predict(lm.fit_reg,data.frame(season="Spring",yr = "y0",mnth ="m2", holiday = "No", hr = "h10",weekday = "w5", weathersit="Clear",temp = 0.24, atemp = 0.2879,hum=0.80,windspeed = 0.2),interval="confidence")
      fit      lwr      upr
1 40.1079 30.76016 49.45563
\end{lstlisting}
Thus 95\% \textbf{Confidence Interval} is [30.76016, 49.45563].
  \item \textit{Is the relationship linear?}\\
  Not all predictors are linear according to our analysis in Section~\ref{sec:non-linear}, therefore we transform some predictors to accommodate non-linear relationships.
  \item \textit{ Is there synergy among the predictors?}\\
  Based on the analysis in Section~\ref{sec:interactive}, between many pairs of predictors there are strong interactions such as \texttt{temp:hum} with $p-value < 2e-16$.
      
\end{enumerate}


\appendix

\section{R commands} 
\label{apd:commands}

\end{document}

