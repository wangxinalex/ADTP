\section{Introduction}
\label{sec:intro} 
Bike sharing systems are new generation of traditional bike rentals where whole process from membership, rental and return
back has become automatic. Through these systems, user is able to easily rent a bike from a particular position and return
back at another position. Currently, there are about over 500 bike-sharing programs around the world which is composed of
over 500 thousands bicycles. Today, there exists great interest in these systems due to their important role in traffic,
environmental and health issues.

Apart from interesting real world applications of bike sharing systems, the characteristics of data being generated by
these systems make them attractive for the research. Opposed to other transport services such as bus or subway, the duration
of travel, departure and arrival position is explicitly recorded in these systems. This feature turns bike sharing system into
a virtual sensor network that can be used for sensing mobility in the city. Hence, it is expected that most of important
events in the city could be detected via monitoring these data.

In this report we will attempt to exploit {\mlr} using R language to construct a linear model. The responses consists of the count of total rental bikes including casual users and registered users and the predictors contains time and weather-related metrics. Based on the model, we attempt to predict the rental count using the given predictor values and provide some suggestions towards the plan of new expansion zones.

The rest of this paper is organized as follows. Section~\ref{sec:technique} provides a small description of the technique used (\mlr), along with its strengths and weaknesses.
Section~\ref{sec:database} relates the data fields appeared in the dataset file provided. Section~\ref{sec:outcome} analyzes the data manipulation as well as the graphs and tables. Section~\ref{sec:model} describes the final suggested model and analyze its performance. Finally, Section~\ref{apd:commands} in appendix will presents all R commands used.